\documentclass{article}
\pagestyle{empty}

% package seperti guna verbatim utk masukkan kode
\usepackage{listings}

\usepackage{tikz}
\usetikzlibrary{arrows, shapes, positioning, calc}

\begin{document}

Insert node between two nodes \newline

\begin{tikzpicture}
  \node  (a) {a}  ;
  \node  (b) at (4,0) {b};
  \path   (a) -- node {m} (b);
  % or \path (a) -- (b) node[midway]{m};
  % or pos =.5  instead of midway
\end{tikzpicture}

\begin{verbatim}
\begin{tikzpicture}
     \node  (a) {a}  ;
     \node  (b) at (4,0) {b};
     \path   (a) -- node {m} (b);
      %  or \path (a) -- (b) node[midway]{m};
      % or pos =.5  instead of midway
\end{tikzpicture}
\end{verbatim}

\hfill\break
Use \verb|calc library| to position \texttt{m} between the two nodes.\newline

\begin{tikzpicture} % with calc library
  \node  (a) {a};
  \node  (b) at (5,1) {b};
  \node   at ($(a)!0.5!(b)$) {m};
\end{tikzpicture}


% guna package listings
\begin{lstlisting}
  \begin{tikzpicture}
    \node  (a) {a};
    \node  (b) at (5,1) {b};
    \node   at ($(a)!0.5!(b)$) {m};
  \end{tikzpicture}
\end{lstlisting}

\hfill\break
Positioning several nodes between two nodes \\

\begin{center}
\begin{tikzpicture}
  \node  (a) {a}  ;
  \node  (b) at (8,0) {b};
  \path (a) -- (b) node[pos=.25]{c} node[pos=.5]{d} node[pos=.75]{e};
\end{tikzpicture}
\end{center}

\begin{lstlisting}
  \node  (a) {a};
  \node  (b) at (8,0) {b};
  \path (a) -- (b) node[pos=.25]{c} node[pos=.5]{d} node[pos=.75]{e};
\end{lstlisting}

\hfill\break
With \verb|positioning| library \verb|\path (a) -- node[below=5] {m} (b);| \\
signifies 5cm, while without signifies 5pt.\newline

\begin{tikzpicture}
  \node (a) {a};
  \node (b) [right= of a] {b};
  \path (a) -- node[below=.5] {m} (b); % distance 0.5cm
\end{tikzpicture}

\end{document}